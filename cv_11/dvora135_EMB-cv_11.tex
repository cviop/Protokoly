\documentclass[a4paper,12pt]{article}   % papír A4, písmo 12 bodu
\usepackage[utf8x]{inputenc}            %kodovaní UTF-8
\usepackage{ucs}                        %kodovani unicode
\usepackage[czech]{babel}               %podpora cestiny
\usepackage[T1]{fontenc}                %pouzij variantu pisma T1 (hacky, carky)
\usepackage[left=2.5cm,right=1.5cm,top=2.5cm,bottom=2.5cm]{geometry} %okraje stranky
\usepackage{amsmath,amsfonts,amssymb}   %podpora matematiky
\usepackage{gensymb,marvosym}           %symboly celsius (\celsius) apod.
%\usepackage{mathptmx}                   %font Times New Roman s~podporou matematiky
\usepackage{times}                      %font Times New Roman (matematika pismem Computer Modern) 
\usepackage{parskip}                    %mezera mezi odstavci
%\usepackage[document]{ragged2e}         %text zarovany vlevo
\usepackage[none]{hyphenat} \sloppy     %slova nedelit a~nepretekat
\usepackage{titlesec}
\setcounter{secnumdepth}{4}
\clubpenalty 10000                      %kontrolovat sirotky
\widowpenalty 10000                     %kontrolovat vdovy
\usepackage{setspace} \onehalfspacing   %podpora pro zmenu radkovani + radkovani 1,5
\usepackage{enumerate}                  %podpora pro zmenu cislovani
\usepackage{fancyhdr}                   %vlastni zahlavi a~zapati
\usepackage{graphicx}                   %podpora grafiky
\graphicspath{{materialy/}}                   %vychozi adresar s~obrazky
\usepackage{caption}                    %popisky
\usepackage{subcaption}                 %podpopisky
\usepackage{siunitx}
\usepackage{MnSymbol,wasysym}
\usepackage[shortlabels]{enumitem}
\usepackage{amsmath}
\usepackage{lastpage}                   %zjištění poslední stránky \pageref{LastPage}
\usepackage{float}                      
\usepackage{url}
\usepackage[unicode]{hyperref}          %klikaci odkazy v~textu
\usepackage{mhchem}
\usepackage{multirow}

\usepackage{halloweenmath}


\titleclass{\subsubsubsection}{straight}[\subsection]
\newcounter{subsubsubsection}[subsubsection]
\renewcommand\thesubsubsubsection{\thesubsubsection.\arabic{subsubsubsection}}
\renewcommand\theparagraph{\thesubsubsubsection.\arabic{paragraph}} % optional, useful if paragraphs are to be numbered


%------------------------ čtvrtá a~pátá úroveň nadpisu ---------------------------

\titleformat{\subsubsubsection}
  {\normalfont\normalsize\bfseries}{\thesubsubsubsection}{1em}{}
\titlespacing*{\subsubsubsection}
{0pt}{3.25ex plus 1ex minus .2ex}{1.5ex plus .2ex}

\makeatletter

\renewcommand\paragraph{\@startsection{paragraph}{5}{\z@}%
  {3.25ex \@plus1ex \@minus.2ex}%
  {-1em}%
  {\normalfont\normalsize\bfseries}}
\renewcommand\subparagraph{\@startsection{subparagraph}{6}{\parindent}%
  {3.25ex \@plus1ex \@minus .2ex}%
  {-1em}%
  {\normalfont\normalsize\bfseries}}
\def\toclevel@subsubsubsection{4}
\def\toclevel@paragraph{5}
\def\toclevel@paragraph{6}
\def\l@subsubsubsection{\@dottedtocline{4}{7em}{4em}}
\def\l@paragraph{\@dottedtocline{5}{10em}{5em}}
\def\l@subparagraph{\@dottedtocline{6}{14em}{6em}}
\makeatother

\setcounter{secnumdepth}{4}
\setcounter{tocdepth}{4}


\setlist[enumerate]{itemsep=0mm}
%_____________________________|___________________________|_____________________________%
%                             |                           |                             %
%-----------------------------| ZDE VYPLNIT UDAJE O PRACI |-----------------------------%
%_____________________________|___________________________|_____________________________%
%                             

\newcommand{\nazev}{Číslicový osciloskop a~generátor programovatelného průběhu}                                                        %
\newcommand{\jmeno}{Jakub Dvořák}                                                     %
\newcommand{\datum}{12.11.2020}                                                              %
%---------------------------------------------------------------------------------------%


%-----------------------------| POUŽITÁ MAKRA |-----------------------------%

%\newcommand{\zkratka}{ve výsledku se mi napíše tenhle text}
%\newcommand{}{}
%\newcommand{}{}
%\newcommand{}{}
\newcommand{\tsub}[1]{$_\textrm{#1}$}
\newcommand{\texp}[1]{$^\textrm{#1}$}
\newcommand{\tohm}{$\Omega$}
\newcommand{\tmu}{$\mu$}


%_______________________________________________________________________________________%
%_______________________________________________________________________________________%


%----------------------------------- KONEC PREAMBULE -----------------------------------%






%-------------------------------------- DOKUMENT --------------------------------------%
%______________________________________________________________________________________%
\begin{document} %%%%%%%%%%%%%%%%%%%%%%%%%%%%%%%%%%%%%%%%%%%%%%%%%%%%%%%%%%%%%%%%%%%%%%%

\setcounter{page}{0} %cislo strany
\pagestyle{empty} %stranku necislovat

%prostredi pro grafy a~schemata \begin{graf} \begin{schema}
\newfloat{schema}{htbp}{schema}\floatname{schema}{Schéma}
\newfloat{graf}{htbp}{graf}\floatname{graf}{Graf}

\begin{titlepage}
    \begin{center}
        \vspace*{1cm}
            
        \Huge
        \textbf{\nazev}
            
        \vspace{0.5cm}
        \LARGE
            
        \vspace{1.5cm}
            
        \textbf{\jmeno}
            
        \vfill
            
        \vspace{0.8cm}
            
        \Large
            
        \datum\\
        \vspace*{.5cm}
        \includegraphics[width=.4\textwidth]{logo-cvut-fee.png}\\
    \end{center}
\end{titlepage}

% --- definice zapati a~cislovani ---
\newpage 
\pagestyle{fancy}                                       %vlastni zahlavi/zapati
\renewcommand{\headrulewidth}{0pt}                      %bez linky v~zahlavi
\renewcommand{\footrulewidth}{.5pt}                    %linka v~zapati - optional
\lhead{}       \chead{} \rhead{\nazev}                        %pole zahlavi (prazdnaV
\lfoot{\jmeno} \cfoot{} \rfoot{\thepage}   %pole zapati


%------------------------------------ VLASTNÍ TEXT ------------------------------------%



\section{Úkol měření}
\label{chap:ukol}
\begin{enumerate}
        \item Seznamte se s~blokovým schématem číslicového osciloskopu na obr. \ref{fig:blok_osc} a~funkčního generátoru na obr. \ref{fig:blok_generator}.
        \item  Odvoďte vztah pro dosažení frekvenčně nezávislého přenosu obvodu pasivní frekvenčně kompenzované sondy 10:1 dle \ref{fig:schema}.
        \item Pomocí generátoru programovatelného průběhu (1 kHz, 1 V~, obdélník) proveďte kompenzaci modelu sondy 10 : 1, která je připojena ke vstupu osciloskopu CH1. Pro obě mezní polohy proměnného kondenzátoru („překompenzovaná“/„podkompenzovaná“ sonda) zakreslete pozorovaný průběh do sešitu a~zjistěte jeho parametry (amplitudu, frekvenci, délku náběžné a~sestupné hrany, příp. překmitnutí). Pro odečítání hodnot těchto parametrů využijte kurzory.
        Na druhém kanálu osciloskopu (vstup CH2) pozorujte a~do sešitu zakreslete průběh téhož signálu snímaný profesionální sondou obdobných parametrů. Tuto sondu nekompenzujte.
        \item Přepněte generátor do režimu„Burst“. Změřte velikost překmitnutí signálu v~případě „překompenzované“ sondy. Pro optimální zobrazení průběhu využijte spouštění sestupnou hranou a~vysvětlete režim osciloskopu „pre-trigger“, ve kterém v~tomto případě přístroj pracuje.
\end{enumerate}


\section{Schéma zapojení}
\label{chap:schema_zapojeni}
\begin{figure}[h!]
  \centering
  \includegraphics[width = 0.8\textwidth]{schema_zapojeni.png}
  \caption{Schéma zapojení}
  \label{fig:schema}
\end{figure}


\begin{figure}[h!]
  \centering
  \includegraphics[width = 0.8\textwidth]{blokove_schema_osciloskop.png}
  \caption{Blokové schéma číslicového osciloskopu}
  \label{fig:blok_osc}
\end{figure}

\begin{figure}[h!]
  \centering
  \includegraphics[width = 0.8\textwidth]{blokove_schema_generator.png}
  \caption{Blokové schéma generátoru programovatelného průběhu}
  \label{fig:blok_generator}
\end{figure}


\section{Seznam použitých přístrojů}
\label{chap:seznam_pristroju}
\begin{table}[H]
  \centering
  \begin{tabular}{ll}
    G&Generátor funkcí Agilent 33220A, rozsah 20~MHz, rozlišení 14~bit, 20~MSa/s\\
    P&Přípravek s~nastavitelným kondenzátorem\\
    SONDA&Profesionální sonda\\
    OSCILOSKOP&Osciloskop DSOX2002A, vstupní impedance 11~M\tohm , 2~pF
  \end{tabular}
\end{table}


\section{Teoretický úvod}
\label{chap:teoreticky_uvod}
Při přenosu signálu dochází k parazitním jevům. Na vstupu osciloskopu najdeme tedy nejen vstupní odpor o hodnotě 1~M\tohm , ale také kapacitu o hodnotě dané výrobcem, zpravidla 15~pF. Při zvyšování frekvence se nepřímo úměrně snižuje kapacitance vnitřního kondenzátoru a~kondenzátoru je poté nižší napětí. Kompenzační kondenzátor v~sondě je určen k tomu, aby se spolu s~kapacitancí v~osciloskopu snižovala i kapacitance v~sondě. Výsledný dělič napětí bude mít v~ideálním případě tedy stále stejný poměr. Pro výpočet hodnoty kapacity kompenzačního kondenzátoru v~sondě můžeme použít následující vzorec:
\begin{equation}
  \frac{R_p}{R_0} = \frac{C_0}{C_p} \rightarrow C_p=\frac{R_O}{R_p}\cdot C_0 = 1,2~pF.
  \label{eq:komp_final}
\end{equation}


\section{Naměřené hodnoty}
\label{chap:namerene_hodnoty}
Naměřené hodnoty jsou zapsány v~tabulce \ref{tab:hodnoty}.

\begin{table}[h!]
  \centering
  \begin{tabular}{|l|c||c||c|}
    \hline
    &Podkompezovaná&Překompenzovaná &Kompenzovaná\\\hline
    Sestupná hrana &15 ns&9 ns&15 ns\\\hline
    Vzestupná hrana&15,3 ns&9 ns&15,3 ns\\\hline
    Překmit&-&575 mV&35 mV\\\hline
  \end{tabular}
  \caption{Naměřené hodnoty}
  \label{tab:hodnoty}
\end{table}

\begin{table}[h!]
  \centering
  \begin{tabular}{|c|c|}
    \hline
    \multicolumn{2}{|c|}{BURST}\\\hline
    překmit&587 mV\\\hline
  \end{tabular}
\end{table}

Níže jsou snímky obrazovky při měření jednotlivých stavů.

\begin{figure}[h!]
  \centering
  \includegraphics[width=.8\textwidth]{overcomp.png}
  \caption{Překompenzovaná sonda}
  \label{fig:overcomp}  
\end{figure}

\begin{figure}[h!]
  \centering
  \includegraphics[width=.8\textwidth]{undercomp.png}
  \caption{Podkompezovaná sonda}
  \label{fig:undercomp}  
\end{figure}

\begin{figure}[h!]
  \centering
  \includegraphics[width=.8\textwidth]{burst.png}
  \caption{Režim Burst}
  \label{fig:burst}  
\end{figure}


\section{Zpracování naměřených hodnot}
\label{chap:zpracovani_hodnot}
Pro výpočet vztahu ze vzorce \ref{eq:komp_final} použijeme následující způsob.

\begin{equation}
  \begin{split}
    \hat{Z_p} &= R_p || \frac{1}{j\omega C_p} ú \frac{\frac{R_p}{j\omega C_p}}{R_p+\frac{1}{j \omega C_p}}\\
    \hat{Z_0} &= R_0 || \frac{1}{j\omega C_0} = \frac{\frac{R_0}{j\omega C_0}}{R_0\frac{1}{j\omega C_O}} = \frac{R_0}{R_0 j\omega C_0 +1}\\
    \frac{\hat{U_2}}{\hat{U_1}} &= \frac{\hat{Z_P}}{\hat{Z_P} + \hat{Z_0}}\\
    \hat{Z_1} &= \frac{R_1}{R_1 j\omega C_1 +1}\\
    \hat{Z_2} &= \frac{R_2}{R_2 j\omega C_2 +1}\\
    \frac{U_2}{U_1} &= \frac{\hat{Z_2}}{\hat{Z_1} + \hat{Z_2}} = \frac{\frac{R_2}{R_2 j\omega C_2 +1}}{\frac{R_2}{R_2 j\omega C_2 +1} + \frac{R_1}{R_1 j\omega C_1 +1}}\\
  \end{split}
\end{equation}
\begin{equation}
  \begin{split}
    R_2\omega j C_2 &= R_1 \omega j C_1\\
    \frac{R_2}{R_1} &= \frac{C_1}{C_2}\\
  \end{split}
\end{equation}


\section{Závěrečné vyhodnocení}
\label{chap:zaver}
Zjistili jsme, že při kompenzaci vstupní kapacity osciloskopu může dojít ke dvěma nechtěným stavům. A to k podkompenzaci a~překompenzaci. Při podkompenzaci má sonda kapacitní charakter a~dochází k integraci signálu, jak je zachyceno na obr. \ref{fig:undercomp} . Při překompenzaci má sonda induktivní charakter, což je vidět na obrázku \ref{fig:overcomp}. Při módu Burst vysílá generátor funkcí $n$ pulzů a~při zvolení režimu \textit{pretrigger} můžeme zachytit nejen změnu náběžné hrany, ale také část signálu. To díky tzv. \textit{kruhovému bufferu}, který ukládá průběh signálu. 


%--- LITERATURA a~ZDROJE (povinne) ---
\clearpage
\renewcommand{\refname}{Seznam použité literatury a~zdrojů informací} 
%\section*{Seznam použité literatury a~zdrojů informací}
\phantomsection %pridej odkaz do PDF zalozek
\addcontentsline{toc}{section}{Seznam použité literatury a~zdrojů informací}

\begin{thebibliography}{99}

%----------------------------------------------------
\subsection*{Seznam použitých internetových zdrojů}
    \bibitem{navod} Návod k~laboratorní úloze
    
\end{thebibliography}

\end{document}
